
\documentclass[12pt,a4paper, twosite]{article}
\usepackage[utf8]{inputenc}
\usepackage[T1]{fontenc}
\usepackage{graphicx}
\usepackage{grffile}
\usepackage{longtable}
\usepackage{wrapfig}
\usepackage{rotating}
\usepackage[normalem]{ulem}
\usepackage{amsmath}
\usepackage{textcomp}
\usepackage{amssymb}
\usepackage{capt-of}
\usepackage{hyperref}
\usepackage[left=2.00cm, right=2.50cm, top=2.50cm, bottom=2.00cm]{geometry}
\usepackage{fancyhdr}
\fancyhead[RO,LE]{\thepage}
\fancyhead[LO]{\emph{\uppercase{\leftmark}}}
\fancyfoot{}
\renewcommand{\headrulewidth}{1.0pt}
\pagestyle{fancy}
\date{}
\title{}
\author{}
\hypersetup{
 pdfauthor={Toledo Perez Cristian Alejandro},
 pdftitle={What is pwa?},
 pdfkeywords={},
 pdfsubject={}
 pdfcreator={Emacs 26.2 (Org mode 9.1.9)}, 
 pdflang={English}}
\begin{document}

\maketitle
\begin{center}
  \textbf{Nombre:} Toledo Perez Cristian Alejandro \\
  \textbf{Grupo:} 10B \\
  \textbf{Materia:} Aplicaciones Web Progresivas \\
  \textbf{Actividad:} Tarea 1: What is pwa? \\
  \textbf{Docente:} Ray Brunet Parra Galaviz \\
  \textbf{Fecha:} 10 de enero de 2024
\end{center}

\newpage
\tableofcontents

\newpage

\section{Introduction}
\label{sec:org60390fa}

In this document, we aim to explore the 
landscape of web-based applications, service-oriented 
applications, native applications, and delve deeper 
into the distinctive characteristics of Progressive Web Apps 
(PWAs). We will discuss their advantages, disadvantages, and 
the underlying technologies that power each type of application. 


\subsection{Purpose}
\label{sec:org434c3ef}

The purpose of this document is to provide 
an in-depth analysis and comparison between 
web applications, service-oriented applications, 
native applications, and Progressive Web Apps. This 
analysis will shed light on their functionalities, 
target platforms, and the user experiences they offer.

\subsection{System Scope}
\label{sec:org12e44a1}

The scope of this document encompasses 
a comprehensive examination of web-based 
applications, service-oriented architecture, 
native applications, and a detailed exploration 
of the key features that define Progressive Web Apps.

\subsection{Acronyms}
\label{sec:orgb158e36}

\begin{itemize}
  \item HTML: HyperText Markup Language
  \item CSS: Cascading Style Sheets
  \item JS: JavaScript
  \item PWA: Progressive Web App
  \item SOA: Service-Oriented Architecture
  \item UX: User Experience
  \item API: Application Programming Interface
  \item GPS: Global Positioning System
  \item JSON: JavaScript Object Notation
  \item HTTPS: Hypertext Transfer Protocol Secure
  \item CPU: Central Processing Unit
  \item RAM: Random Access Memory
  \item UI: User Interface
  \item CLI: Command Line Interface
  \item SDK: Software Development Kit
  \item IDE: Integrated Development Environment
\end{itemize}

\subsection{References}
\label{sec:org62711e0}

\begin{thebibliography}
{9}

\bibitem{Gillis, A. S.} Gillis, A. S. (2022, December 8)  native app. Software Quality.January 9, 2024, from \url{https://www.techtarget.com/searchsoftwarequality/definition/native-application-native-app}

\bibitem{Fernández, C.} Fernández, C. (2022, March 18). Cross-platform app development. Characteristics. ABAMobile. January 9, 2024, from \url{https://abamobile.com/web/cross-platform-app-development-characteristics/}

\bibitem{AWS} aws (n.d.). what is SOA? - SOA architecture explained. Amazon Web Services, Inc. January 9, 2024, from \url{https://aws.amazon.com/what-is/service-oriented-architecture/}

\bibitem{Vue Storefront} vuestorefront (n.d.). what is PWA? Progressive Web Apps Explained | Vue Storefront. Vue Storefront. January 9, 2024, from \url{https://vuestorefront.io/blog/pwa}

\end{thebibliography}


\subsection{Overview of the Document}
\label{sec:orgdaca22c}

The document is structured to provide 
insights into different types of applications 
prevalent in modern software development. 
We will delve into the specifics of each category, 
their strengths, weaknesses, and their implications 
in today's digital landscape.

\newpage

\section{What is PWA?}
\label{sec:orgc1c4017}

A Progressive Web App (PWA) is an application developed 
using web technologies such as HTML, CSS, and JavaScript, 
aiming to offer a seamless and engaging User Experience (UX) 
similar to that of a native application. PWAs leverage modern 
web capabilities to function across various platforms and devices, 
employing a single codebase.

\subsection{Concepts from pwa}
\label{sec:org24980a8}

\begin{itemize}
  \item Service Workers:
  Service workers represent a fundamental element within PWAs. These are JavaScript-based scripts operating discreetly in the background, decoupled from the web page. Their primary role encompasses managing tasks like caching, push notifications, and offline functionality. Their implementation ensures the PWA's robustness, enabling reliable operation, even in scenarios with limited or no network connectivity.

  \item Web App Manifest:
  The web app manifest stands as a JSON file providing crucial metadata about the PWA, comprising details such as its name, description, icons, and designated start URL. This file serves to instruct the browser on how to treat the web app, allowing it to be perceived as an installable application and supporting features such as home screen addition.

  \item Responsive Design:
  Responsive design constitutes an integral aspect ensuring the adaptability and aesthetic appeal of PWAs across a spectrum of devices and screen sizes. Commonly utilizing CSS media queries and flexible layouts, this feature guarantees an optimal visual experience.

  \item HTTPS:
  The utilization of a secure HTTPS connection is a mandatory requirement for PWAs. It serves as a protective measure ensuring the confidentiality and integrity of data exchanged between users and servers. HTTPS is pivotal for enabling critical PWA functionalities like service workers and push notifications.

  \item App Shell Architecture:
  The app shell denotes the minimal HTML, CSS, and JavaScript essential for rendering the foundational user interface of the PWA. Its prompt loading facilitates a swift and dependable initial user experience, with subsequent content loaded dynamically as required.

  \item Caching Strategies:
  The implementation of caching mechanisms is a cornerstone of PWAs, facilitating offline functionality and expedited loading times. Developers employ caching strategies through service workers to store and retrieve assets like images, stylesheets, and scripts.

  \item Offline Capabilities:
  Designed to ensure uninterrupted user experiences, PWAs adeptly operate even in offline or suboptimal network conditions. This resilience is achieved through the strategic utilization of service workers and caching strategies, enabling seamless functionality sans an active internet connection.

  \item Push Notifications:
  Enabling real-time updates and user engagement, push notifications are a pivotal feature of PWAs. Service workers play a key role in managing and displaying these notifications, ensuring user interaction even when the app is not actively accessed.

  \item IndexedDB and Local Storage:
  Leveraging technologies such as IndexedDB and local storage, PWAs efficiently store data on the client side. This capability enables data persistence and empowers offline functionality.

  \item Cross-Browser Compatibility:
  PWAs are meticulously crafted for compatibility across diverse browsers. Developers adhere to standardized web APIs and features universally supported by modern browsers, ensuring seamless performance irrespective of the browser being utilized.

  \item Automatic Updates:
  PWAs exhibit the capability to autonomously update, ensuring users consistently access the latest version without necessitating manual intervention. This self-updating functionality is executed through service workers and a sophisticated cache update strategy.
\end{itemize}

\subsection{What’s the difference between Progressive Web Apps and Native Mobile Apps?}
\label{sec:orgaf51da6}


Native apps have dominated for a decade, but 
there is now a shift towards a unified experience 
across all platforms and devices with Progressive Web Apps 
(PWAs). PWAs are mobile-first oriented and significantly lighter 
compared to native apps, while still delivering a high level of 
interactivity.
Although major players like Alibaba, Pinterest, and 
Twitter have already embraced and implemented PWAs on their 
sites, the term "Progressive Web App" remained somewhat vague 
for smaller entities for years. Many of them, recognizing the 
importance of a mobile-first approach, have been convinced of the 
necessity of Responsive Web Design or native apps
However, awareness of the advantages of PWAs is steadily 
increasing, as evidenced by Google's statistics and the growing 
number of live projects using Vue Storefront. This suggests a 
shifting perception towards these progressive applications.

\newpage

\section{SOA (Service-Oriented Architecture)}
\label{sec:org40573d1}

Service-Oriented Architecture (SOA) stands as a 
strategic methodology in software development, 
employing program elements termed "services" to 
craft robust business applications. SOA's primary intent 
lies in the reusability of services across diverse systems 
or their aggregation to execute intricate tasks. Operating 
as an architectural paradigm, SOA enables the construction 
of applications through a consortium of services fostering 
seamless communication. This interaction could encompass simple 
data transmission or orchestrate collaborative activities among 
two or more services. Establishing interconnections among these 
services remains imperative for SOA's functionality.

\subsection{advantages}
\label{sec:orgb8b6b9e}
SOA has the following advantages:
\begin{itemize}
\item Service Reusability: Services can be used across various applications, reducing redundancy and improving development efficiency.
\item Flexibility: Easily adapts to changes in business requirements by modifying or adding services.
\item Interoperability: Facilitates communication between heterogeneous systems, enabling the integration of diverse technologies.
\item Scalability: Can be easily scaled by adding new services or replicating existing services as needed.
\item Simplified Maintenance: Modifying one service does not impact other parts of the system, facilitating maintenance.
\end{itemize}


\subsection{disadvantages}
\label{sec:orgb8b6b9a}
SOA has the following disadvantages:
\begin{itemize}

\item Initial Complexity: Implementing SOA can be complex and require careful planning, potentially increasing initial costs.
\item Version Management: Managing service versions can become complicated, especially in environments with multiple applications and services.
\item Performance: Communication between services may introduce some latency, affecting performance compared to monolithic architectures.
\item Security: Managing security can be challenging, as service communication often occurs over networks.
\end{itemize}


\subsection{Characteristics of SOA}
\label{sec:org94bc543}

\begin{itemize}
  \item Reusability: Services can be reused in different applications and contexts, facilitating development efficiency.
  \item Interoperability: Enables effective communication between different services and applications, even if developed on different platforms.
  \item Service Discovery: Services can be easily discovered and accessed, facilitating their integration into new applications.
  \item Abstraction: Services are designed to be independent of underlying implementations, allowing changes in implementation without affecting the service interface.
  \item Composition: Allows the combination of services to create more complex and functional applications.
\end{itemize}


\newpage


\section{Native app and Cross-platform}
\label{sec:orgb8b6b9b}

\subsection{Native app}
\label{sec:orgb8b6b9c}
A native application refers to a tailored 
software application meticulously crafted by 
developers for a distinct platform. Tailored 
specifically for a particular device and its operating 
system, native apps harness the distinct hardware and 
software attributes of the device. These apps offer optimized 
performance, leveraging the device's unique capabilities, such 
as GPS integration, in contrast to generic web or mobile cloud 
applications designed to cater to multiple systems.

\subsection{Native app characteristics}
\label{sec:orgb8b6b2b}

\begin{itemize}
  \item Tailored for a specific platform like Android, iOS, Windows, or Blackberry.
  \item Developed using a programming language specific to the targeted platform.
  \item Utilizes the latest mobile device technology, including GPS and camera features.
  \item Available for download and installation from designated app stores like the Apple App Store or Google Play Store.
  \item Capable of functioning offline without requiring an internet connection.
\end{itemize}

\subsection{Cross-platform}
\label{sec:orgb8b6b9d}
Cross-platform software stands as a 
versatile application compatible with 
multiple operating systems. It isn't bound 
to a singular platform or device, running 
seamlessly across Windows, Mac OS X, and Linux 
environments. Often known as multi-platform software 
or platform-independent software, cross-platform applications 
exhibit the following characteristics:

\newpage


\section{Appendices}
\label{sec:org75cea03}


\end{document}